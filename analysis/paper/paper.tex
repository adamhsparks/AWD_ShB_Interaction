% Template for PLoS
% Version 3.4 January 2017
\documentclass[10pt,letterpaper]{article}
\usepackage[top=0.85in,left=2.75in,footskip=0.75in]{geometry}

% amsmath and amssymb packages, useful for mathematical formulas and symbols
\usepackage{amsmath,amssymb}

% Use adjustwidth environment to exceed column width (see example table in text)
\usepackage{changepage}

% Use Unicode characters when possible
\usepackage[utf8x]{inputenc}

% textcomp package and marvosym package for additional characters
\usepackage{textcomp,marvosym}

% cite package, to clean up citations in the main text. Do not remove.
% \usepackage{cite}

% Use nameref to cite supporting information files (see Supporting Information section for more info)
\usepackage{nameref,hyperref}

% line numbers
\usepackage[right]{lineno}

% ligatures disabled
\usepackage{microtype}
\DisableLigatures[f]{encoding = *, family = * }

% color can be used to apply background shading to table cells only
\usepackage[table]{xcolor}

% array package and thick rules for tables
\usepackage{array}

% create "+" rule type for thick vertical lines
\newcolumntype{+}{!{\vrule width 2pt}}

% create \thickcline for thick horizontal lines of variable length
\newlength\savedwidth
\newcommand\thickcline[1]{%
  \noalign{\global\savedwidth\arrayrulewidth\global\arrayrulewidth 2pt}%
  \cline{#1}%
  \noalign{\vskip\arrayrulewidth}%
  \noalign{\global\arrayrulewidth\savedwidth}%
}

% \thickhline command for thick horizontal lines that span the table
\newcommand\thickhline{\noalign{\global\savedwidth\arrayrulewidth\global\arrayrulewidth 2pt}%
\hline
\noalign{\global\arrayrulewidth\savedwidth}}


% Remove comment for double spacing
%\usepackage{setspace} 
%\doublespacing

% Text layout
\raggedright
\setlength{\parindent}{0.5cm}
\textwidth 5.25in 
\textheight 8.75in

% Bold the 'Figure #' in the caption and separate it from the title/caption with a period
% Captions will be left justified
\usepackage[aboveskip=1pt,labelfont=bf,labelsep=period,justification=raggedright,singlelinecheck=off]{caption}
\renewcommand{\figurename}{Fig}

% Use the PLoS provided BiBTeX style
% \bibliographystyle{plos2015}

% Remove brackets from numbering in List of References
\makeatletter
\renewcommand{\@biblabel}[1]{\quad#1.}
\makeatother

% Leave date blank
\date{}

% Header and Footer with logo
\usepackage{lastpage,fancyhdr,graphicx}
\usepackage{epstopdf}
\pagestyle{myheadings}
\pagestyle{fancy}
\fancyhf{}
\setlength{\headheight}{27.023pt}
\lhead{\includegraphics[width=2.0in]{PLOS-submission.eps}}
\rfoot{\thepage/\pageref{LastPage}}
\renewcommand{\footrule}{\hrule height 2pt \vspace{2mm}}
\fancyheadoffset[L]{2.25in}
\fancyfootoffset[L]{2.25in}
\lfoot{\sf PLOS}

%% Include all macros below
\newcommand{\lorem}{{\bf LOREM}}
\newcommand{\ipsum}{{\bf IPSUM}}





\usepackage{forarray}
\usepackage{xstring}
\newcommand{\getIndex}[2]{
  \ForEach{,}{\IfEq{#1}{\thislevelitem}{\number\thislevelcount\ExitForEach}{}}{#2}
}

\setcounter{secnumdepth}{0}

\newcommand{\getAff}[1]{
  \getIndex{#1}{University of Southern Queensland,International Rice Research Institute}
}

\providecommand{\tightlist}{%
  \setlength{\itemsep}{0pt}\setlength{\parskip}{0pt}}

\begin{document}
\vspace*{0.2in}

% Title must be 250 characters or less.
\begin{flushleft}
{\Large
\textbf\newline{Do Alternate Wetting and Drying Irrigation Technology and Nitrogen Rates
Affect Rice Insect Populations and Diseases?} % Please use "sentence case" for title and headings (capitalize only the first word in a title (or heading), the first word in a subtitle (or subheading), and any proper nouns).
}
\newline
\\
A.H. Sparks\textsuperscript{\getAff{University of Southern Queensland}}\textsuperscript{*},
N.P. Castilla\textsuperscript{\getAff{International Rice Research Institute}},
B.A.R. Hadi\textsuperscript{\getAff{International Rice Research Institute}},
B.O. Sander\textsuperscript{\getAff{International Rice Research Institute}}\\
\bigskip
\textbf{\getAff{University of Southern Queensland}}Centre for Crop Health, West Street, Toowoomba, Queensland, Australia,
4350\\
\textbf{\getAff{International Rice Research Institute}}Los Baños, Laguna, Philippines, 4031\\
\bigskip
* Corresponding author: adam.sparks@usq.edu.au\\
\end{flushleft}
% Please keep the abstract below 300 words
\section*{Abstract}
Water and nitrogen management play vital roles in rice production.
However, the mismanagement of these two management practices may trigger
sheath blight of rice, caused by \emph{Rhizoctonia solani}, which is
favored by wet conditions, high relative humidity, and high nitrogen
fertilizer levels. To understand how different combinations of water and
nitrogen management affect sheath blight epidemics, we conducted two
separate split-plot experiments with a water saving (alternate wetting
and drying) regime and traditional flood irrigation regime combined with
differing nitrogen treatments in the dry seasons of 2015 and 2016.
Disease was scored in the same way in both experiments using a sheath
blight assessment scale for field evaluation developed at the
International Rice Research Institute to assess the severity on infected
sheaths and leaves while sheath blight incidence on tillers were counted
per hill. We were unable to detect any differences in disease in either
experiment due to irrigation regime, N rates or the interaction of the
two treatments in either season. This suggests that farmers can adopt
water saving technologies without risking increased sheath blight
incidence. We suggest that further cross-cutting research in this area
is warranted.

% Please keep the Author Summary between 150 and 200 words
% Use first person. PLOS ONE authors please skip this step. 
% Author Summary not valid for PLOS ONE submissions.   
\section*{Author summary}
Lorem ipsum dolor sit amet, consectetur adipiscing elit. Curabitur eget
porta erat. Morbi consectetur est vel gravida pretium. Suspendisse ut
dui eu ante cursus gravida non sed sem. Nullam sapien tellus, commodo id
velit id, eleifend volutpat quam. Phasellus mauris velit, dapibus
finibus elementum vel, pulvinar non tellus. Nunc pellentesque pretium
diam, quis maximus dolor faucibus id. Nunc convallis sodales ante, ut
ullamcorper est egestas vitae. Nam sit amet enim ultrices, ultrices elit
pulvinar, volutpat risus.

\linenumbers

% Use "Eq" instead of "Equation" for equation citations.
\emph{Text based on plos sample manuscript, see
\url{http://journals.plos.org/ploscompbiol/s/latex}}

\section{Introduction}\label{introduction}

Alternate wetting and drying (AWD) is an irrigation technique for
irrigated rice developed by the International Rice Research Institute
(IRRI) and its partners that saves about 15-40\% of irrigation water
{[}1,2{]}. In AWD rice fields are exposed to several dry phases during
the growth period without exposing the plants to water stress. In order
to avoid yield decline under AWD ``safe'' thresholds have been
developed. Under safe AWD irrigation water is applied when the field
water level reaches 15cm below the soil surface {[}3{]}. Fields are
furthermore kept flooded during the flowering period. Besides saving
water AWD also reduces greenhouse gas (GHG) emissions of rice fields,
which is a substantial factor in the GHG budget of rice producing
countries, by around 50\% {[}4,5{]}.

The AWD technology has been identified as promising climate smart
practice for different rice growing regions that can stabilize rice
production in water scarce areas as well as help reduce the carbon
footprint of rice production. Various countries, e.g., Bangladesh,
Vietnam, Thailand and the Philippines, plan to widely apply AWD to local
rice production {[}6{]}. However, a change in water regime in rice
fields on large scale might encompass different other effects, for
example related to plant health.

We therefore established field experiments in order to\ldots{}

\section{Materials and Methods}\label{materials-and-methods}

\subsection{Experimental design}\label{experimental-design}

Two experiments were conducted at the International Rice Research
Institute's (IRRI) Ziegler Experiment Station in Los Baños, Calabarzon,
Philippines (latitude 14°11' N, , longitude 121°15' E) in the 2015 and
2016 dry seasons. For the 2016 season changes were made to optimize the
experiment based on findings from the 2015 season. Both seasons
consisted of split plot design with four replicates where irrigation was
the main plot and nitrogen (N) rate was the split plot treatment. The
changes between seasons and experiments are detailed following.

\subsubsection{2015 Dry Season}\label{dry-season}

The main plot size was 12m x 12m (144 sq m), with a sub-plot size of 5m
x 5m (25 sq m). Replication size was 12m x 24m (288 sq m) with a buffer
of 1m per sub plot for a whole experiment size of 1,152 sq m. The main
plot treatments were alternate wetting and drying (AWD) and ) and
continuously flooded (CF) or farmers' practice.

Irrigation in AWD plots was determined by the water level in plots,
i.e., when the water level reached 15cm below the soil surface
irrigation water was applied to a level of 5cm. In CF plots a standing
water layer of 3-5cm was maintained throughout the growing season.

The sub plot treatments were different rates of nitrogen\ldots{}

\subsubsection{2016 Dry Season}\label{dry-season-1}

In 2016 dry season the plot size was increased and due to these changes,
the sizes of the replicates are not equal as necessitated by the use of
a larger area for the experiment. The main plot sizes were: Block 1 (B1)
21m x 20.5m (412.5 sq m) and Block 2 (B2) 20.25m x 21.6m (437.4 sq m).
The sub plot sizes were B1 21m x 10.25m (215.25 sq m), B2 20.25m x 10.8m
(218.7 sq m). The replication sizes were B1 - 42m x 20.5m (861 sq m) and
B2 - 40.5m x 21.6m (874.8 sq m). A buffer 0.5m per sub plot was used and
the overall experiment size was 3471.6 sq m.

\subsection{Data collection and
analysis}\label{data-collection-and-analysis}

Data were analysed using multivariate generalised linear mixed models
implemented in the MCMCglmm package {[}7{]} in R {[}8{]}.

\section{Results}\label{results}

\section{Discussion}\label{discussion}

\section{Conclusions (optional)}\label{conclusions-optional}

\section{Acknowledgments}\label{acknowledgments}

\section*{References}\label{references}
\addcontentsline{toc}{section}{References}

Supporting information captions (if applicable)

\hypertarget{refs}{}
\hypertarget{ref-Bouman2001}{}
1. Bouman B, Tuong T. Field water management to save water and increase
its productivity in irrigated lowland rice. Agricultural Water
Management. 2001;49: 11--30.
doi:\href{https://doi.org/https://doi.org/10.1016/S0378-3774(00)00128-1}{https://doi.org/10.1016/S0378-3774(00)00128-1}

\hypertarget{ref-Feng2007}{}
2. Feng L, Bouman B, Tuong T, Cabangon R, Li Y, Lu G, et al. Exploring
options to grow rice using less water in northern china using a
modelling approach: I. field experiments and model evaluation.
Agricultural Water Management. 2007;88: 1--13.
doi:\href{https://doi.org/https://doi.org/10.1016/j.agwat.2006.10.006}{https://doi.org/10.1016/j.agwat.2006.10.006}

\hypertarget{ref-Richards2014}{}
3. Richards M, Sander BO. Alternate wetting and drying in irrigated
rice. Copenhagen, Denmark: CGIAR Research Program on Climate Change,
Agriculture; Food Security (CCAFS); 2014 Apr.

\hypertarget{ref-Yan2005}{}
4. Yan X, Yagi K, Akiyama H, Akimoto H. Statistical analysis of the
major variables controlling methane emission from rice fields. Global
Change Biology. Blackwell Science Ltd; 2005;11: 1131--1141.
doi:\href{https://doi.org/10.1111/j.1365-2486.2005.00976.x}{10.1111/j.1365-2486.2005.00976.x}

\hypertarget{ref-Sander2015}{}
5. Sander BO, Wassmann R, Siopongco JDLC. Mitigating greenhouse gas
emissions from rice production through water-saving techniques:
Potential, adoption and empirical evidence. In: Hoanh CT, Johnston R,
Smakhtin V, editors. Centre for Agriculture; Biosciences International;
2016. p. 193.

\hypertarget{ref-MOEF2015}{}
6. Environment M of, Government of the People's Republic of Bangladesh F
(MOEF). Intended nationally determined contributions (indc).

\hypertarget{ref-MCMCglmm2017}{}
7. Hadfield JD. MCMC methods for multi-response generalized linear mixed
models: The MCMCglmm R package. Journal of Statistical Software.
2010;33: 1--22. Available: \url{http://www.jstatsoft.org/v33/i02/}

\hypertarget{ref-R2017}{}
8. R Core Team. R: A language and environment for statistical computing
{[}Internet{]}. Vienna, Austria: R Foundation for Statistical Computing;
2017. Available: \url{https://www.R-project.org/}

\nolinenumbers


\end{document}

